\documentclass{article}
\usepackage{proof}
\usepackage{bussproofs}
\usepackage{xcolor}
\usepackage{url}
\usepackage{framed}
\usepackage{amsmath, amsfonts, amssymb}
\usepackage{mathtools}

\begin{document}
\title{Survey of ITP Hammer Tools}
\author{Arjun Viswanathan}
\date{}
\maketitle
\begin{abstract}
	Interactive theorem provers (ITPs) and automatic theorem provers (ATPs)
	are two distinct categories of theorem provers with some significant 
	trade-offs between the two. On the one hand, ITPs are typically robust
	with a small, verified kernel resulting in a high degree of reliability.
	However, they require user intervention in the proving process, only 
	offering a minimal amount of automation. ATPs on the other hand are 
	push-button theorem provers that use complex heuristics to prove 
	theorems; as a consequence, they have a large code-base that is hard 
	to maintain and prone to bugs. A lot of recent research has focused on 
	bridging the gap between these two poles of theorem proving, quite often 
	in the form of a tool called a hammer, whose goal - among other 
	things - is to increase the automation in ITPs with the help of 
	ATPs. This work aims to survey some popular hammer tools.
\end{abstract}

\section{Introduction}
\label{sec:intro}
	Interactive theorem provers (ITPs) or proof assistants are 
	software tools that allow formalizing of mathematical proofs.
	They provide an expressive logic to state theorems in, and 
	an interactive interface through which the user can 
	attempt to prove these theorems using programs typically 
	called tactics. Generally, this interface mimics a 
	written mathematical proof with a context and goal 
	that changes on the fly as you step through the parts 
	of the proof. The structures provided by the ITP are 
	minimal and the user's mathematical structures are only 
	defined on top of these, keeping the verified kernel of the 
	ITP small. These proofs provide a high level of reliability
	but are hard to come up with from the user's point of view. 
	 



\section{Formal Preliminaries}
\label{sec:prelim}



\section{Comparison}
\label{sec:comp}



\section{Conclusion and Moving Forward}
\label{sec:conc}

\bibliographystyle{abbrv}
\bibliography{bib}

\end{document}