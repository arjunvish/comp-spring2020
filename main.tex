\documentclass{article}
\usepackage{proof}
\usepackage{bussproofs}
\usepackage{xcolor}
\usepackage{url}
\usepackage{framed}
\usepackage{amsmath, amsfonts, amssymb}
\usepackage{mathtools}

\begin{document}
\title{Hammers and Certified Checkers}
\author{Arjun Viswanathan}
\date{}
\maketitle
\begin{abstract}
	Interactive theorem provers (ITPs) and automatic theorem provers (ATPs)
	are two distinct categories of theorem provers on different ends 
	of the spectrum of theorem provers. On the one hand, 
	ITPs are typically robust tools with a small, verified kernel 
	allowing for a high degree of reliability. However, they 
	require user intervention in the proving process, only
	offering a minimal amount of automation. ATPs, on the other hand, 
	are push-button theorem provers that use complex heuristics to prove 
	theorems; as a consequence, they have a large code-base that is hard 
	to maintain and susceptible to bugs. A lot of recent research 
	has focused on bridging the gap between these two poles 
	of theorem proving. Hammers and certified checkers are tools
	that were born from this research, that have different 
	approaches to this problem. This work aims to 
	comprehensively study the differences between these 
	tools from the point of view of using SMT solvers as
	the ATPs enhancing automation in ITPs.
\end{abstract}

\section{Introduction}
\label{sec:intro}
	Interactive theorem provers (ITPs) or proof assistants are 
	software tools that allow formalizing of mathematical proofs.
	They provide an expressive logic to state theorems in, and 
	an interactive interface through which the user can 
	attempt to prove these theorems using programs typically 
	called tactics. Generally, this interface mimics a 
	written mathematical proof with a context and goal 
	that changes on the fly as one steps through the parts 
	of the proof. The structures provided by the ITP are 
	minimal and the user's mathematical structures are 
	defined on top of these, keeping the verified kernel of the 
	ITP small. These proofs provide a high level of reliability
	but are hard to come up with from the user's point of view. 
	
	Automatic theorem provers (ATPs) have grown rapidly over the 
	past decades and refer to tools that allow automatic proving 
	of logical formulas. Interaction between the user and the 
	ATP is kept to a minimum; ideally, the user would provide a 
	theorem to the ATP and the ATP either proves it or comes up
	with a counter-example that disproves it. For this work, we 
	will look at a particular sub-category of ATPs called 
	satisfiability modulo theories (SMT)-solvers. SMT solvers 
	input formulas in first order logic with equality and 
	other background theories such as (integer, real, and 
	machine integer) arithmetic, arrays, etc and determine
	whether the formula is satisfiable or not. Depending 
	on the theory, and depending on whether quantification
	is enabled, the SMT-solver could also not answer the query. 
	This satisfiability problem is the dual of the 
	validity problem (proving a formula to be valid) - a 
	formula can be proved to be valid by establishing that 
	its negation is unsatisfiable. Although ATPs can also 
	refer to superposition provers such as Vampire and SPASS, 
	which have also been successfully used to provide 
	automation to ITPs, the focus here is on leveraging 
	SMT-solvers in ITPs.



\section{Formal Preliminaries}
\label{sec:prelim}



\section{Comparison}
\label{sec:comp}



\section{Conclusion and Moving Forward}
\label{sec:conc}

\bibliographystyle{abbrv}
\bibliography{bib}

\end{document}