\documentclass{article}
\usepackage{proof}
\usepackage{bussproofs}
\usepackage{xcolor}
\usepackage{url}
\usepackage{framed}
\usepackage{amsmath, amsfonts, amssymb}
\usepackage{mathtools}

\begin{document}
\title{SMT Solving for the Theory of Bit-Vectors}
\author{Arjun Viswanathan}
\date{}
\maketitle
\begin{abstract}
	Satisfiability modulo theories (SMT) is the problem
	of satisfying a first-order formula with theories such as integer, real, and bit-vector arithmetic, strings, 
	sets, etc. This work focuses on the theory of 
	bit-vector arithmetic and the state of the art for 
	solving satisfiability of both quantified and 
	quantifier-free bit-vector formulas. Bit-vectors 
	are sequences of 0/1 bits, frequently used to 
	represent and verify properties about 
	finite-precision integers. Due to their closeness
	to hardware, they are also used in hardware
	verification.
\end{abstract}

\section{Introduction}
\label{sec:intro}
	\textcolor{red}{Introduce SAT solving and SMT solving}
	\textcolor{red}{Introduce bit-vector solving with an 
	example}
	\textcolor{red}{Say something about quantifiers}
	\textcolor{red}{Talk about state-of-the-art solvers 
	for BV solving}
	\textcolor{red}{Give a preview of the rest of the paper here}

\section{Formal Preliminaries}
\label{sec:prelim}
	\textcolor{red}{Maybe introduce a running example?}

\section{Quantifier-Free Fragment}
\label{sec:qfbv}
	Bit-blasting and binary decision diagrams. 
	Model constructign calculus. Stochastic local search.
	Boolector uses SLS + bit-blasting.
	Lazy vs Eager solving?
	~\cite{10.1007/978-3-642-22438-6_11} This is how you cite. 
	
\section{Quantified Bit-Vectors}
\label{sec:bv}
	Model-Based Quantifier Instantiation and CEGQI. 
	
\section{Using BVs to Reason about Z's}
\label{sec:bv2Z}
	SMTEN way.
	Bedrock encoding.
	SMTRAT.
	
\section{Separate Calculus for Different BV Ops?}
\label{sec:calc}

	
	
\section{To-Do}
	\begin{enumerate}
		\item Write some shit.
	\end{enumerate}
	
	
\section{Conclusion and Moving Forward}
\label{sec:conc}

\bibliographystyle{abbrv}
\bibliography{bib}

\end{document}